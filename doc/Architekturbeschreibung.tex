\documentclass[fontsize=12pt,paper=a4,twoside]{scrartcl}

% SWP-Präambel
% C 2003-2017 Sebastian Offermann, Rainer Koschke, Karsten Hölscher
% In Zeilen 40 und 41 sind jeweils die aktuellen Daten einzutragen

\usepackage[utf8]{inputenc}     % Kodierung der Tex-Datei
\usepackage[T1]{fontenc}        % Korrekte Ausgabe von Sonderzeichen (Umlaute)
\usepackage[ngerman]{babel}     % Deutsche Einstellungen [ab \begin{document}]

\usepackage{bibgerm}            % Bibliographie
\usepackage{fancyhdr}           % obere Seitenränder gestalten
\usepackage{float}              % Floats Objekte mit [H] festsetzen
\usepackage{graphicx}           % Graphiken als jpg, png etc. einbinden
\usepackage{moreverb}           % zusätzliche verbatim-Umgebungen
\usepackage{pdflscape}          % PDF-Support für landscape
\usepackage[final]{pdfpages}    % Externe PDFs einbinden
\usepackage{stmaryrd}           % zusätzliche Symbole
\usepackage{supertabular}       % Tabellen über Seitenränder hinaus
\usepackage{tabularx}           % Tabellen mit vorgegebener Breite
\usepackage{url}                % setzt URLs schön mit \url{http://bla.laber.com/~mypage}

%%% Die Reihenfolge der folgenden Pakete muss beibehalten werden:
%%% varioref, hyperref, cleveref, bookmark
% Verweise innerhalb des Dokuments schick mit " ... auf Seite ... "
% automatisch versehen. Dazu \vref{labelname} benutzen
\usepackage[ngerman]{varioref}  % [vor hyperref für korrekte Verweise]
\usepackage[colorlinks=true, pdfstartview=FitV, linkcolor=blue,
            citecolor=blue, urlcolor=blue, hyperfigures=true,
            pdftex=true]{hyperref} % [vor bookmark wegen der Optionen]
\usepackage[ngerman]{cleveref}
\usepackage{bookmark}

\hyphenation{Arbeits-paket}     % Trennungsregeln

%%% Definitionen
\newcommand{\grad}{\ensuremath{^{\circ}} }
\renewcommand{\strut}{\vrule width 0pt height5mm depth2mm}
\newcommand{\gq}[1]{\glqq{}#1\grqq{}}

%%% Semesterkonstanten
\newboolean{langversion} %Deklaration
\setboolean{langversion}{true} %Zuweisung ist 'false' für Blockkurs
\newcommand{\jahr}[1]{2020} %2017/2018

% erstes Argument: SWP-2, zweites SWP-1
\newcommand{\highlight}[1]{\textcolor{blue}{\textbf{#1}}}
\newcommand{\variante}[2]{\ifthenelse{\boolean{langversion}}{#1}{#2}}
\newcommand{\nurlangversion}[0]%
    {\variante{\highlight{}}%Muss in SWP-2 ausgefüllt werden}}%
              {\highlight{Entfällt in SWP-1}}}
\newcommand{\swp}[0]{Software-Projekt \variante{2}{1}}
\newcommand{\semester}[0]{SoSe \jahr}

%%% Formatierungsanpassungen
% Damit Latex nicht zu lange Zeilen produziert:
\sloppy
%Uneinheitlicher unterer Seitenrand:
%\raggedbottom

% Kein Erstzeileneinzug beim Absatzanfang
% Sieht aber nur gut aus, wenn man zwischen Absätzen viel Platz einbaut
\setlength{\parindent}{0ex}

% Abstand zwischen zwei Absätzen
\setlength{\parskip}{1ex}

% Seitenränder für Korrekturen verändern
\addtolength{\evensidemargin}{-1cm}
\addtolength{\oddsidemargin}{1cm}

\bibliographystyle{gerapali}

% 1. Parameter: Euer/Eure TutorIn, z. B. {Kim Harrison}
% 2. Parameter: Abgabedatum, z. B. {05. April 2063}
% 3. Parameter: Versionsnummer, z. B. {1.1}
% 4.-9. Parameter: jeweils Name und (Uni-)Email-Adresse jedes 
%                 Gruppenmitglieds; mit einem & getrennt, z. B.
% {Robin Cowl & roco@tzi.de}
% Besteht die Gruppe aus weniger als 6 Personen, so werden die 
% übrigen Parameter leer gelassen: {}
\newcommand \swpdocument[9] {
% Lustige Header auf den Seiten
  \pagestyle{fancy}
  \setlength{\headheight}{70.55003pt}
  \fancyhead{}
  \fancyhead[LO,RE]{\swp{}\\%
                    \semester{}\\%
                    \documentTitle}
  \fancyhead[LE,RO]{Seite \thepage\\%
                    \slshape \leftmark\\%
                    \slshape \rightmark}

% Lustige Header nur auf dieser Seite (Titelseite)
  \thispagestyle{fancy}
  \fancyhead[LO,RE]{ }
  \fancyhead[LE,RO]{Universität Bremen\\%
                    FB 3 -- Informatik\\%
                    Dr. Karsten Hölscher\\%
                    TutorIn: #1}
  \fancyfoot[C]{}

% Start Titelseite
  \vspace{3cm}
  \begin{minipage}[H]{\textwidth}
    \begin{center}
      \bfseries \Large \swp{} -- \semester{}\\
      \smallskip
      \small VAK 03-BA-901.02\\
      \vspace{3cm}
    \end{center}
  \end{minipage}
  \begin{minipage}[H]{\textwidth}
    \begin{center}
      \vspace{1cm}
      \bfseries \Large \documentTitle\\
      \vfill
    \end{center}
  \end{minipage}
  \vfill
  \begin{minipage}[H]{\textwidth}
    \begin{center}
      \sffamily
      \begin{tabular}{lr}
        #4 \\
        #5 \\
        #6 \\
        #7 \\
        #8 \\
        #9 \\
      \end{tabular}
      \\[22mm]
      \itshape Abgabe: #2 --- Version #3 \\ ~
    \end{center}
  \end{minipage}
% Ende Titelseite

% Start Inhaltsverzeichnis
\newpage
  \thispagestyle{fancy}
  \fancyhead{}
  \fancyhead[LO,RE]{\swp{}\\%
                    \semester{}\\%
                    \documentTitle}
  \fancyhead[LE,RO]{Seite \thepage\\%
                    \slshape \leftmark\\~}
  \fancyfoot{}
  \renewcommand{\headrulewidth}{0.4pt}
  \tableofcontents
% Ende Inhaltsverzeichnis

% Header für alle weiteren Seiten
\newpage
  \fancyhead[LE,RO]{Seite \thepage\\%
                    \slshape \leftmark\\%
                    \slshape \rightmark}

}



%
% Und jetzt geht das Dokument los....
%
\begin{document}
\newcommand\documentTitle{Architekturbeschreibung}
\swpdocument{Karsten Hölscher}{TT. Monat JJJJ}{1.1}%
            {Fabian Kehlenbeck & fkehlenb@tzi.de}%
            {Leonard Haddad & s\_xsipo6@tzi.de}%
            {Luca Nittscher & lnittsch@tzi.de}%
            {Rasmus Burwitz & rburwitz@tzi.de}%
            {Samuel Nejati Masouleh & samnej@tzi.de}%
            {Aaron Rudkowski & rudkowsk@tzi.de}%

%%%%%%%%%%%%%%%%%%%%%%%%%%%%%%%%%%%%%%%%%%%%%%%%%%%%%%%%%%%%%%%%%%%%%%%%
\section*{Version und Änderungsgeschichte}

{\em Die aktuelle Versionsnummer des Dokumentes sollte eindeutig und gut zu
identifizieren sein, hier und optimalerweise auf dem Titelblatt.}

\begin{tabular}{ccl}
Version & Datum & Änderungen \\
\hline
0.1 & TT.MM.JJJJ & Dokumentvorlage als initiale Fassung kopiert \\
0.2 & TT.MM.JJJJ & \ldots \\
\ldots
\end{tabular}


%%%%%%%%%%%%%%%%%%%%%%%%%%%%%%%%%%%%%%%%%%%%%%%%%%%%%%%%%%%%%%%%%%%%%%%%
\section{Einführung}

\subsection{Zweck}

{ \em Was ist der Zweck dieser Architekturbeschreibung? Wer sind die LeserInnen?}

\subsection{Status}
  
\subsection{Definitionen, Akronyme und Abkürzungen}

\subsection{Referenzen}

\subsection{Übersicht über das Dokument}


%%%%%%%%%%%%%%%%%%%%%%%%%%%%%%%%%%%%%%%%%%%%%%%%%%%%%%%%%%%%%%%%%%%%%%%%
\section{Globale Analyse} \label{sec:globale_analyse}

{\itshape Hier werden Einflussfaktoren aufgezählt und bewertet sowie Strategien
zum Umgang mit interferierenden Einflussfaktoren entwickelt.}

\subsection{Einflussfaktoren} \label{sec:einflussfaktoren}
{\itshape Hier sind Einflussfaktoren gefragt, die sich auf die Architektur
beziehen. Es sind ausschließlich architekturrelevante Einflussfaktoren, und 
nicht z.\,B.\ solche, die lediglich einen Einfluss auf das Projektmanagement 
haben. Fragt Euch also bei jedem Faktor: Beeinflusst er wirklich die 
Architektur? Macht einen einfachen Test: Wie würde die Architektur aussehen, 
wenn ihr den Einflussfaktor $E$ berücksichtigt? Wie würde sie aussehen, wenn Ihr
$E$ nicht berücksichtigt? Kommt in beiden Fällen dieselbe Architektur heraus, 
dann kann der Einflussfaktor nicht architekturrelevant sein.

Es geht hier um Einflussfaktoren, die
\begin{enumerate}
  \item sich über die Zeit ändern,
  \item viele Komponenten betreffen,
  \item schwer zu erfüllen sind oder
  \item mit denen man wenig Erfahrung hat.
\end{enumerate}

Die Flexibilität und Veränderlichkeit müssen ebenfalls charakterisiert werden. 
\begin{enumerate}
  \item Flexibilität: Könnt Ihr den Faktor zum jetzigen Zeitpunkt beeinflussen?
  \item Veränderlichkeit: ändert der Faktor sich später durch äußere Einflüsse?
\end{enumerate}

Unter Auswirkungen sollte dann beschrieben werden, \emph{wie} der Faktor 
\emph{was} beeinflusst. Das können sein:
\begin{itemize}
  \item andere Faktoren
  \item Komponenten
  \item Operationsmodi
  \item Designentscheidungen (Strategien)
\end{itemize}

Verwendet eine eindeutige Nummerierung der Faktoren, um sie auf den 
Problemkarten einfach referenzieren zu können.}

\subsection{Probleme und Strategien} \label{sec:strategien}

{\itshape Aus einer Menge von Faktoren ergeben sich Probleme, die nun in Form 
von Problemkarten beschrieben werden. Diese resultieren z.\,B.\ aus
\begin{itemize}
  \item Grenzen oder Einschränkungen durch Faktoren
  \item der Notwendigkeit, die Auswirkung eines Faktors zu begrenzen
  \item der Schwierigkeit, einen Produktfaktor zu erfüllen, oder
  \item der Notwendigkeit einer allgemeinen Lösung zu globalen Anforderungen.
\end{itemize}
Dazu entwickelt Ihr Strategien, um mit den identifizierten Problemen umzugehen.

Achtet auch hier darauf, dass die Probleme und Strategien wirklich die 
Architektur betreffen und nicht etwa das Projektmanagement. Die Strategien 
stellen im Prinzip die Designentscheidungen dar. Sie sollten also die Erklärung 
für den konkreten Aufbau der verschiedenen Sichten liefern.

Beschreibt möglichst mehrere Alternativen und gebt an, für welche Ihr Euch 
letztlich aus welchem Grunde entschieden habt. Natürlich müssen die genannten 
Strategien in den folgenden Sichten auch tatsächlich umgesetzt werden!

Ein sehr häufiger Fehler ist es, dass SWP-Gruppen arbeitsteilig vorgehen: die 
eine Gruppe schreibt das Kapitel zur Analyse von Faktoren und zu den Strategien, 
die andere Gruppe beschreibt die diversen Sichten, ohne dass diese beiden 
Gruppen sich abstimmen. Natürlich besteht aber ein Zusammenhang zwischen den
Faktoren, Strategien und Sichten. Dieser muss erkennbar sein, indem sich die 
verschiedenen Kapitel eindeutig aufeinander beziehen.}


%%%%%%%%%%%%%%%%%%%%%%%%%%%%%%%%%%%%%%%%%%%%%%%%%%%%%%%%%%%%%%%%%%%%%%%%
\section{Konzeptionelle Sicht} \label{sec:konzeptionell}

{\itshape Diese Sicht beschreibt das System auf einer hohen Abstraktionsebene,
d.\,h. mit sehr starkem Bezug zur Anwendungsdomäne und den geforderten
Produktfunktionen und "~attributen. Sie legt die Grobstruktur fest, ohne gleich 
in die Details von spezifischen Technologien abzugleiten. Sie wird in den 
nachfolgenden Sichten konkretisiert und verfeinert. Die konzeptionelle Sicht 
wird mit {UML}-Komponentendiagrammen visualisiert.}


%%%%%%%%%%%%%%%%%%%%%%%%%%%%%%%%%%%%%%%%%%%%%%%%%%%%%%%%%%%%%%%%%%%%%%%%
\section{Modulsicht} \label{sec:modulsicht}

{\itshape Diese Sicht beschreibt den statischen Aufbau des Systems mit Hilfe von
Modulen, Subsystemen, Schichten und Schnittstellen. Diese Sicht ist 
hierarchisch, d.\,h. Module werden in Teilmodule zerlegt. Die Zerlegung endet 
bei Modulen, die ein klar umrissenes Arbeitspaket für eine Person darstellen und
in einer Kalenderwoche implementiert werden können. Die Modulbeschreibung der 
Blätter dieser Hierarchie muss genau genug und ausreichend sein, um das Modul 
implementieren zu können.

Die Modulsicht wird durch {UML}-Paket- und Klassendiagramme visualisiert.

Die Module werden durch ihre Schnittstellen beschrieben.
Die Schnittstelle eines Moduls $M$ ist die Menge aller Annahmen, die andere 
Module über $M$ machen dürfen, bzw.\ jene Annahmen, die $M$ über seine 
verwendeten Module macht (bzw. seine Umgebung, wozu auch Speicher, Laufzeit 
etc.\ gehören).
Konkrete Implementierungen dieser Schnittstellen sind das Geheimnis des Moduls
und können vom Programmierer festgelegt werden. Sie sollen hier dementsprechend 
nicht beschrieben werden. 

Die Diagramme der Modulsicht sollten die zur Schnittstelle gehörenden Methoden
enthalten. Die Beschreibung der einzelnen Methoden (im Sinne der 
Schnittstellenbeschreibung) geschieht allerdings per Javadoc im zugehörigen 
Quelltext. Das bedeutet, dass Ihr für alle Eure Module Klassen, Interfaces und 
Pakete erstellt und sie mit den Methoden der Schnittstellen verseht. Natürlich 
noch ohne Methodenrümpfe bzw.\ mit minimalen Rümpfen. Dieses Vorgehen 
vereinfacht den Schnittstellenentwurf und stellt Konsistenz sicher.

Jeder Schnittstelle liegt ein Protokoll zugrunde. Das Protokoll beschreibt die 
Vor- und Nachbedingungen der Schnittstellenelemente. Dazu gehören die erlaubten
Reihenfolgen, in denen Methoden der Schnittstelle aufgerufen werden dürfen, 
sowie Annahmen über Eingabeparameter und Zusicherungen über Ausgabeparameter. 
Das Protokoll von Modulen wird in der Modulsicht beschrieben.
Dort, wo es sinnvoll ist, sollte es mit Hilfe von Zustands- oder 
Sequenzdiagrammen spezifiziert werden. Diese sind dann einzusetzen, wenn der
Text allein kein ausreichendes Verständnis vermittelt (insbesondere bei 
komplexen oder nicht offensichtlichen Zusammenhängen).

Der Bezug zur konzeptionellen Sicht muss klar ersichtlich sein. Im Zweifel 
sollte er explizit erklärt werden. Auch für diese Sicht muss die Entstehung 
anhand der Strategien erläutert werden.}


%%%%%%%%%%%%%%%%%%%%%%%%%%%%%%%%%%%%%%%%%%%%%%%%%%%%%%%%%%%%%%%%%%%%%%%%
\section{Datensicht} \label{sec:datensicht}

{\itshape Hier wird das der Anwendung zugrundeliegende Datenmodell beschrieben. 
Hierzu werden neben einem erläuternden Text auch ein oder mehrere 
{UML}-Klassendiagramme verwendet. Das hier beschriebene Datenmodell wird u.\,a.\ 
jenes der Anforderungsspezifikation enthalten, allerdings mit 
implementierungsspezifischen Änderungen und Erweiterungen. Siehe die gesonderten
Hinweise.}


%%%%%%%%%%%%%%%%%%%%%%%%%%%%%%%%%%%%%%%%%%%%%%%%%%%%%%%%%%%%%%%%%%%%%%%%
\section{Ausführungssicht} \label{sec:ausfuehrung}

{\itshape Die Ausführungssicht beschreibt das Laufzeitverhalten. Hier werden die
Laufzeitelemente aufgeführt und beschrieben, welche Module sie zur Ausführung 
bringen. Ein Modul kann von mehreren Laufzeitelementen zur Laufzeit verwendet 
werden. Die Ausführungssicht beschreibt darüber hinaus, welche Laufzeitelemente 
spezifisch miteinander kommunizieren. Zudem wird bei verteilten Systemen 
(z.\,B.\ Client-Server-Systeme) dargestellt, welche Module von welchen Prozessen
auf welchen Rechnern ausgeführt werden.}


%%%%%%%%%%%%%%%%%%%%%%%%%%%%%%%%%%%%%%%%%%%%%%%%%%%%%%%%%%%%%%%%%%%%%%%%
\section{Zusammenhänge zwischen Anwendungsfällen und Architektur}
\sectionmark{Zusammenhänge AF u. Architektur} \label{sec:anwendungsfaelle}

{\itshape In diesem Abschnitt sollen Sequenzdiagramme mit Beschreibung(!) für 
\variante{zwei bis drei von Euch ausgewählte Anwendungsfälle}%
{einen von Euch ausgewählten Anwendungsfall}
erstellt werden. Ein Sequenzdiagramm beschreibt den Nachrichtenverkehr zwischen 
allen Modulen, die an der Realisierung des Anwendungsfalles beteiligt sind. 
\variante{Wählt die Anwendungsfälle so, dass nach Möglichkeit alle Module Eures
entworfenen Systems in mindestens einem Sequenzdiagramm vorkommen. Falls Euch 
das nicht gelingt, versucht möglichst viele und die wichtigsten Module 
abzudecken.}%
{Dazu könnt ihr Euch einen Anwendungsfall heraussuchen, der möglichst viele 
Module der  Architektur abdeckt. In SWP-2 werden wir mehrere Anwendungsfälle
betrachten und eine umfangreichere Abdeckung der Architektur anstreben.} }


%%%%%%%%%%%%%%%%%%%%%%%%%%%%%%%%%%%%%%%%%%%%%%%%%%%%%%%%%%%%%%%%%%%%%%%%
\section{Evolution} \label{sec:evolution}

{\itshape Beschreibt in diesem Abschnitt, welche Änderungen Ihr vornehmen müsst,
wenn sich Anforderungen oder Rahmenbedingungen ändern. Insbesondere würden 
hierbei die in der Anforderungsspezifikation unter \glqq{}Ausblick\grqq{} 
genannten Punkte behandelt werden.}

\dots

\end{document}

%%% Local Variables: 
%%% mode: latex
%%% mode: reftex
%%% mode: flyspell
%%% ispell-local-dictionary: "de_DE"
%%% TeX-master: t
%%% End: 
