\documentclass[fontsize=12pt,paper=a4,twoside]{scrartcl}

\usepackage{longtable}


\usepackage[normalem]{ulem}
\useunder{\uline}{\ul}{}

% SWP-Präambel
% C 2003-2017 Sebastian Offermann, Rainer Koschke, Karsten Hölscher
% In Zeilen 40 und 41 sind jeweils die aktuellen Daten einzutragen

\usepackage[utf8]{inputenc}     % Kodierung der Tex-Datei
\usepackage[T1]{fontenc}        % Korrekte Ausgabe von Sonderzeichen (Umlaute)
\usepackage[ngerman]{babel}     % Deutsche Einstellungen [ab \begin{document}]

\usepackage{bibgerm}            % Bibliographie
\usepackage{fancyhdr}           % obere Seitenränder gestalten
\usepackage{float}              % Floats Objekte mit [H] festsetzen
\usepackage{graphicx}           % Graphiken als jpg, png etc. einbinden
\usepackage{moreverb}           % zusätzliche verbatim-Umgebungen
\usepackage{pdflscape}          % PDF-Support für landscape
\usepackage[final]{pdfpages}    % Externe PDFs einbinden
\usepackage{stmaryrd}           % zusätzliche Symbole
\usepackage{supertabular}       % Tabellen über Seitenränder hinaus
\usepackage{tabularx}           % Tabellen mit vorgegebener Breite
\usepackage{url}                % setzt URLs schön mit \url{http://bla.laber.com/~mypage}

%%% Die Reihenfolge der folgenden Pakete muss beibehalten werden:
%%% varioref, hyperref, cleveref, bookmark
% Verweise innerhalb des Dokuments schick mit " ... auf Seite ... "
% automatisch versehen. Dazu \vref{labelname} benutzen
\usepackage[ngerman]{varioref}  % [vor hyperref für korrekte Verweise]
\usepackage[colorlinks=true, pdfstartview=FitV, linkcolor=blue,
            citecolor=blue, urlcolor=blue, hyperfigures=true,
            pdftex=true]{hyperref} % [vor bookmark wegen der Optionen]
\usepackage[ngerman]{cleveref}
\usepackage{bookmark}

\hyphenation{Arbeits-paket}     % Trennungsregeln

%%% Definitionen
\newcommand{\grad}{\ensuremath{^{\circ}} }
\renewcommand{\strut}{\vrule width 0pt height5mm depth2mm}
\newcommand{\gq}[1]{\glqq{}#1\grqq{}}

%%% Semesterkonstanten
\newboolean{langversion} %Deklaration
\setboolean{langversion}{true} %Zuweisung ist 'false' für Blockkurs
\newcommand{\jahr}[1]{2020} %2017/2018

% erstes Argument: SWP-2, zweites SWP-1
\newcommand{\highlight}[1]{\textcolor{blue}{\textbf{#1}}}
\newcommand{\variante}[2]{\ifthenelse{\boolean{langversion}}{#1}{#2}}
\newcommand{\nurlangversion}[0]%
    {\variante{\highlight{}}%Muss in SWP-2 ausgefüllt werden}}%
              {\highlight{Entfällt in SWP-1}}}
\newcommand{\swp}[0]{Software-Projekt \variante{2}{1}}
\newcommand{\semester}[0]{SoSe \jahr}

%%% Formatierungsanpassungen
% Damit Latex nicht zu lange Zeilen produziert:
\sloppy
%Uneinheitlicher unterer Seitenrand:
%\raggedbottom

% Kein Erstzeileneinzug beim Absatzanfang
% Sieht aber nur gut aus, wenn man zwischen Absätzen viel Platz einbaut
\setlength{\parindent}{0ex}

% Abstand zwischen zwei Absätzen
\setlength{\parskip}{1ex}

% Seitenränder für Korrekturen verändern
\addtolength{\evensidemargin}{-1cm}
\addtolength{\oddsidemargin}{1cm}

\bibliographystyle{gerapali}

% 1. Parameter: Euer/Eure TutorIn, z. B. {Kim Harrison}
% 2. Parameter: Abgabedatum, z. B. {05. April 2063}
% 3. Parameter: Versionsnummer, z. B. {1.1}
% 4.-9. Parameter: jeweils Name und (Uni-)Email-Adresse jedes 
%                 Gruppenmitglieds; mit einem & getrennt, z. B.
% {Robin Cowl & roco@tzi.de}
% Besteht die Gruppe aus weniger als 6 Personen, so werden die 
% übrigen Parameter leer gelassen: {}
\newcommand \swpdocument[9] {
% Lustige Header auf den Seiten
  \pagestyle{fancy}
  \setlength{\headheight}{70.55003pt}
  \fancyhead{}
  \fancyhead[LO,RE]{\swp{}\\%
                    \semester{}\\%
                    \documentTitle}
  \fancyhead[LE,RO]{Seite \thepage\\%
                    \slshape \leftmark\\%
                    \slshape \rightmark}

% Lustige Header nur auf dieser Seite (Titelseite)
  \thispagestyle{fancy}
  \fancyhead[LO,RE]{ }
  \fancyhead[LE,RO]{Universität Bremen\\%
                    FB 3 -- Informatik\\%
                    Dr. Karsten Hölscher\\%
                    TutorIn: #1}
  \fancyfoot[C]{}

% Start Titelseite
  \vspace{3cm}
  \begin{minipage}[H]{\textwidth}
    \begin{center}
      \bfseries \Large \swp{} -- \semester{}\\
      \smallskip
      \small VAK 03-BA-901.02\\
      \vspace{3cm}
    \end{center}
  \end{minipage}
  \begin{minipage}[H]{\textwidth}
    \begin{center}
      \vspace{1cm}
      \bfseries \Large \documentTitle\\
      \vfill
    \end{center}
  \end{minipage}
  \vfill
  \begin{minipage}[H]{\textwidth}
    \begin{center}
      \sffamily
      \begin{tabular}{lr}
        #4 \\
        #5 \\
        #6 \\
        #7 \\
        #8 \\
        #9 \\
      \end{tabular}
      \\[22mm]
      \itshape Abgabe: #2 --- Version #3 \\ ~
    \end{center}
  \end{minipage}
% Ende Titelseite

% Start Inhaltsverzeichnis
\newpage
  \thispagestyle{fancy}
  \fancyhead{}
  \fancyhead[LO,RE]{\swp{}\\%
                    \semester{}\\%
                    \documentTitle}
  \fancyhead[LE,RO]{Seite \thepage\\%
                    \slshape \leftmark\\~}
  \fancyfoot{}
  \renewcommand{\headrulewidth}{0.4pt}
  \tableofcontents
% Ende Inhaltsverzeichnis

% Header für alle weiteren Seiten
\newpage
  \fancyhead[LE,RO]{Seite \thepage\\%
                    \slshape \leftmark\\%
                    \slshape \rightmark}

}



%
% Und jetzt geht das Dokument los....
%
\begin{document}
\newcommand\documentTitle{Architekturbeschreibung}
\swpdocument{Karsten Hölscher}{TT. Monat JJJJ}{1.1}%
            {Fabian Kehlenbeck & fkehlenb@tzi.de}%
            {Leonard Haddad & s\_xsipo6@tzi.de}%
            {Luca Nittscher & lnittsch@tzi.de}%
            {Rasmus Burwitz & rburwitz@tzi.de}%
            {Samuel Nejati Masouleh & samnej@tzi.de}%
            {Aaron Rudkowski & rudkowsk@tzi.de}%

%%%%%%%%%%%%%%%%%%%%%%%%%%%%%%%%%%%%%%%%%%%%%%%%%%%%%%%%%%%%%%%%%%%%%%%%
\section*{Version und Änderungsgeschichte}

{\em Die aktuelle Versionsnummer des Dokumentes sollte eindeutig und gut zu
identifizieren sein, hier und optimalerweise auf dem Titelblatt.}

\begin{tabular}{ccl}
Version & Datum & Änderungen \\
\hline
0.1 & TT.MM.JJJJ & Dokumentvorlage als initiale Fassung kopiert \\
0.2 & TT.MM.JJJJ & \ldots \\
\ldots
\end{tabular}


%%%%%%%%%%%%%%%%%%%%%%%%%%%%%%%%%%%%%%%%%%%%%%%%%%%%%%%%%%%%%%%%%%%%%%%%
\section{Einführung}

\subsection{Zweck}

{ \em Was ist der Zweck dieser Architekturbeschreibung? Wer sind die LeserInnen?}

\subsection{Status}
  
\subsection{Definitionen, Akronyme und Abkürzungen}

\subsection{Referenzen}

\subsection{Übersicht über das Dokument}


%%%%%%%%%%%%%%%%%%%%%%%%%%%%%%%%%%%%%%%%%%%%%%%%%%%%%%%%%%%%%%%%%%%%%%%%
\section{Globale Analyse} \label{sec:globale_analyse}

%{\itshape Hier werden Einflussfaktoren aufgezählt und bewertet sowie Strategien
%zum Umgang mit interferierenden Einflussfaktoren entwickelt.}

%\subsection{Einflussfaktoren} \label{sec:einflussfaktoren}
%{\itshape Hier sind Einflussfaktoren gefragt, die sich auf die Architektur
%beziehen. Es sind ausschließlich architekturrelevante Einflussfaktoren, und 
%nicht z.\,B.\ solche, die lediglich einen Einfluss auf das Projektmanagement 
%haben. Fragt Euch also bei jedem Faktor: Beeinflusst er wirklich die 
%Architektur? Macht einen einfachen Test: Wie würde die Architektur aussehen, 
%wenn ihr den Einflussfaktor $E$ berücksichtigt? Wie würde sie aussehen, wenn Ihr
%$E$ nicht berücksichtigt? Kommt in beiden Fällen dieselbe Architektur heraus, 
%dann kann der Einflussfaktor nicht architekturrelevant sein.

%Es geht hier um Einflussfaktoren, die
%\begin{enumerate}
%  \item sich über die Zeit ändern,
%  \item viele Komponenten betreffen,
%  \item schwer zu erfüllen sind oder
%  \item mit denen man wenig Erfahrung hat.
%\end{enumerate}

%Die Flexibilität und Veränderlichkeit müssen ebenfalls charakterisiert werden. 
%\begin{enumerate}
%  \item Flexibilität: Könnt Ihr den Faktor zum jetzigen Zeitpunkt beeinflussen?
%  \item Veränderlichkeit: ändert der Faktor sich später durch äußere Einflüsse?
%\end{enumerate}

%Unter Auswirkungen sollte dann beschrieben werden, \emph{wie} der Faktor 
%\emph{was} beeinflusst. Das können sein:
%\begin{itemize}
%  \item andere Faktoren
%  \item Komponenten
%  \item Operationsmodi
%  \item Designentscheidungen (Strategien)
%\end{itemize}

%Verwendet eine eindeutige Nummerierung der Faktoren, um sie auf den 
%Problemkarten einfach referenzieren zu können.}

\begin{longtable}[c]{|p{5cm}|p{5cm}|p{5cm}|}
\hline
\multicolumn{1}{|c|}{\textbf{Einflussfaktor}} & \multicolumn{1}{c|}{\textbf{Veränderlichkeit/Flexibilität}} & \multicolumn{1}{c|}{\textbf{Auswirkungen}}  \\ \hline
\endhead
%ORGANISATION
\multicolumn{3}{|l|}{{\textbf{O1: Organisation}}} 
\\ \hline
\multicolumn{3}{|l|}{{O1.1: Teamgröße}} 
\\ \hline
Die Gruppe besteht aus 6 Entwicklern. & Keine Flexibilität, aber Veränderlichkeit: Eine oder mehrere Personen könnten die Gruppe verlassen      &  Die Architektur kann wegen Zeitmangel und fehlenden Fähigkeiten Mangel enthalten.
\\ \hline 
\multicolumn{3}{|l|}{{O1.2: Architekturabgabe}} 
\\ \hline
Die Architektur soll am 31.05.2020 abgegeben werden. & Keine Flexibilität, oder Veränderlichkeit.  & Eventuell gibt es Mängel in der Architektur, die uns erst in der Implementierung auffallen. 
\\ \hline             
\multicolumn{3}{|l|}{{O1.3: Endabgabe}} 
\\ \hline
Das Endprodukt muss am 02.08.2020 abgegeben werden. & Keine Flexibilität, oder Veränderlichkeit.   &  Eventuell können nicht alle geplanten Funktionen implementiert werden.
\\ \hline
\multicolumn{3}{|l|}{{O1.4: Fähigkeiten}} 
\\ \hline
Die Gruppenmitglieder haben unterschiedliche Programmiererfahrungen und Kentnisse. & Keine Flexibilität, aber Veränderlichkeit: Neues Wissen kann sich angeeignet werden.  & Die Implementierung kann Mangel enthalten. 
\\ \hline     
%TECHNIK
\multicolumn{3}{|l|}{{\textbf{T1: Technik}}} 
\\ \hline 
\multicolumn{3}{|l|}{{T1.1: Sprache}} 
\\ \hline
Es soll Java 8 oder höher verwendet werden. & Eine gewisse Flexiblität und Veränderlichkeit in der Java Version, allerdings keine in der Sprache an sich.  & Das Projekt muss in Java umgesetzt werden. 
\\ \hline                 
\multicolumn{3}{|l|}{{T1.2: Plattformen}} 
\\ \hline
Das Spiel soll unter Linux, MacOS und Windows lauffähig sein & Keine Veränderlichkeit oder Flexiblität  & Bei der Implementierung muss darauf geachtet werden, plattformunabhängig vorzugehen. 
\\ \hline
\multicolumn{3}{|l|}{{T1.3: Datenbank}} 
\\ \hline
Es soll eine leichtgewichtige, relationale Datenbank verwendet werden. & Große Flexibilität und Veränderlichkeit: Da es keine festen Vorgaben gibt, ist die Datenbank relativ frei wählbar. Allerdings muss eine Datenbank verwendet werden. & Wir müssen uns für eine Datenbank entscheiden, und diese bei der Implementierung verwenden. 
\\ \hline
\multicolumn{3}{|l|}{{T1.4: libgdx}} 
\\ \hline
Es soll libdgx verwendet werden. & Keine Veränderlichkeit oder Flexibilität.   & Bei der Implementierung muss libgdx verwendet werden. 
\\ \hline
\multicolumn{3}{|l|}{{T1.5: Client}} 
\\ \hline
Das Spiel soll ohne weitere Installationen auf Clientseite spielbar sein. & Keine Flexibilität, oder Veränderlichkeit.    &  Bei der Implementierung muss darauf geachtet werden, keine Abhängigkeiten von Dingen zu haben, die auf Clientseite weitere Downloads erfordern. 
\\ \hline
\multicolumn{3}{|l|}{{T1.6: Server}} 
\\ \hline
Es soll eine Spielserver geben, der von mindestens zwei Spieler*innen benutzt werden kann. & Keine Flexibilität, oder Veränderlichkeit.    & Wir müssen einen Spielserver einrichten, der das Spiel kontrolliert. 
\\ \hline
%PRODUKT
\multicolumn{3}{|l|}{{\textbf{Produktfaktoren}}} 
\\ \hline
%
\multicolumn{3}{|l|}{{\textbf{P1: Sektionen}}} 
\\ \hline
\multicolumn{3}{|l|}{{P1.1: Sektionen}} 
\\ \hline
Das Raumschiff soll in unterschiedliche Sektionen unterteilt sein. & Keine Flexibilität, oder Veränderlichkeit.    & Die Architektur muss vorsehen, dass ein Raumschiff eine Ansammlung an Sektionen ist. 
\\ \hline 
\multicolumn{3}{|l|}{{P1.2: Systeme}} 
\\ \hline
Jede Sektion soll relevante Systeme haben.  & Keine Flexibilität, oder Veränderlichkeit.    & Die Architektur muss unterschiedlichen Sektionen die relevanten Systeme zuordnen. 
\\ \hline
\multicolumn{3}{|l|}{{P1.3: Beschädigungen}} 
\\ \hline
Sektionen sollen im Kampf beschädigt werden können.  & Keine Flexibilität, oder Veränderlichkeit.    & Die Architektur muss vorsehen, dass die Sektionen unabhängig voneinander beschädigt werden können. Ebenfalls muss für jede Sektion gespeichert und angezeigt werden, wie heile/kaputt sie ist. 
\\ \hline
\multicolumn{3}{|l|}{{P1.4: Systembeschädigungen}} 
\\ \hline
Wenn eine Sektion beschädigt ist, soll dies auch die enthaltenen Systeme beeinflussen. & Keine Flexibilität, oder Veränderlichkeit.    & Die Architektur muss vorsehen, dass für jedes System der Status gespeichert wird. 
\\ \hline
%
\multicolumn{3}{|l|}{{\textbf{P2: Ressourcen}}} 
\\ \hline
\multicolumn{3}{|l|}{{P2.1: Ressourcen}} 
\\ \hline
Das Raumschiff soll verschieden Ressourcen haben, wie Geld, Energie, Status Außenhülle.  & Keine Flexibilität, oder Veränderlichkeit.    & Die Architektur muss für das Raumschiff übergeordnete Ressourcen speichern. 
\\ \hline                                                      
\multicolumn{3}{|l|}{{P2.2: Ressourcenveränderlichkeit}} 
\\ \hline
Ressourcen sollen veränderlich sein. & Keine Flexibilität, oder Veränderlichkeit.    & Die Architektur muss vorsehen, dass die Eigenschaften nicht konstant sind, und durch Methoden auf die Werte einfluss genommen werden kann. 
\\ \hline
\multicolumn{3}{|l|}{{P2.3: Ressourcenverfügung}} 
\\ \hline
Ressourcen sollen pro Spielrunde zur Verfügung stehen. & Keine Flexibilität, oder Veränderlichkeit.    & Die Architektur muss vorsehen, dass der Spieler in jeder Runde Zugriff auf die Ressourcen hat und diese beeinflussen kann. 
\\ \hline
%
\multicolumn{3}{|l|}{{\textbf{P3: Eigenschaften}}} 
\\ \hline
\multicolumn{3}{|l|}{{P3.1: Eigenschaften}} 
\\ \hline
Ein Raumschiff soll verschiedene Eigenschaften haben, wie Waffen, Schutzschilde, Hüllenpanzerungen, und Energie. & Keine Flexibilität, oder Veränderlichkeit.    &  Die Architektur muss für das Raumschiff übergeordnete Eigenschaften vorsehen, die für alle Sektionen gelten. 
\\ \hline
\multicolumn{3}{|l|}{{P3.2: Waffen}} 
\\ \hline
Es sollen mehrere Waffen pro Raumschiff möglich sein. & Keine Flexibilität, oder Veränderlichkeit.    & Die Architektur muss vorsehen, dass für ein Raumschiff mehrere Waffen gespeichert und eingesetzt werden können, die Spieler*in muss also auch die Waffe auswählen, mit der sie schießen möchte. 
\\ \hline
\multicolumn{3}{|l|}{{P3.3: Waffenunterschiede}} 
\\ \hline
Unterschiedliche Waffen sollen sich mindestens in Ladezeit, Trefferwahrscheinlichkeit und Schadenshöhe unterscheiden. & Keine Flexibilität, oder Veränderlichkeit.    &  In der Architektur muss vorgesehen sein, dass mindestens diese Werte keine Konstanten sind, und das Spiel muss damit umgehen können, in diesen Attributen unterschiedliche Werte zu haben. 
\\ \hline
\multicolumn{3}{|l|}{{P3.4: Verbesserung von Eigenschaften}} 
\\ \hline
Eigenschaften sollen durch Geld verbessert werden können. & Keine Flexibilität, oder Veränderlichkeit.    & Die Architektur muss vorsehen, dass die Eigenschaften veränderlich sein sollen. Ebenfalls muss es Möglichkeiten geben, bei denen der Spieler Geld durch Verbesserungen austauschen kann.  
\\ \hline
%
\multicolumn{3}{|l|}{{\textbf{P4: Besatzung}}} 
\\ \hline
\multicolumn{3}{|l|}{{P4.1: Besatzung}} 
\\ \hline
Das Raumschiff soll eine Besatzung haben. & Keine Flexibilität, oder Veränderlichkeit.    &  Die Architektur muss vorsehen, dass für ein Raumschiff gespeichert wird, welche Besatzungsmitglieder an Bord sind. 
\\ \hline  
\multicolumn{3}{|l|}{{P4.2: Aufenthaltsorte Besatzung}} 
\\ \hline
Die Besatzungsmitglieder sollen sich in Sektionen aufhalten können. & Keine Flexibilität, oder Veränderlichkeit.    &  Die Architektur muss vorsehen, dass gespeichert wird, in welcher Sektion sich welches Mitglied aufhält. 
\\ \hline                                                  
\multicolumn{3}{|l|}{{P4.3: Besatzung und Systeme}} 
\\ \hline
Die Systeme einer Sektion sollen durch die Besatzungsmitglieder beeinflusst werden können, zum Beispiel sollen Systeme/Sektionen durch die Besatzung repariert werden können. & Keine Flexibilität, oder Veränderlichkeit.    &  Die Architektur muss vorsehen, dass die Besatzung mit den Sektionen interagieren kann, entweder automatisch durch ihren Aufenthalt, oder durch explizite Kommandos des Spielers. 
\\ \hline
\multicolumn{3}{|l|}{{P4.4: Tod}} 
\\ \hline
Besatzungsmitglieder sollen sterben können. & Keine Flexibilität, oder Veränderlichkeit.    & Die Architektur muss vorsehen, dass Besatzungsmitglieder einen Gesundheitsstatus haben, der sinken kann. Ebenfalls sollten Mitgleider entweder gelöscht oder inaktiviert werden, sobald sie sterben. 
\\ \hline
\multicolumn{3}{|l|}{{P4.5: Neue Besatzungsmitglieder}} 
\\ \hline
Neue Besatzungsmitglieder sollen auf das Schiff aufgenommen werden können. & Keine Flexibilität, oder Veränderlichkeit.    &  Die Architektur muss vorsehen, dass die Besatzung nicht konstant ist. 
\\ \hline
\multicolumn{3}{|l|}{{P4.6: Besatzungsmitglieder Fähigkeiten}} 
\\ \hline
Besatzungsmitglieder sollen Fähigkeiten haben. Diese sollen sich auf Systeme auswirken, falls sich das Mitglied in der entsprechenden Sektion befindet. & Keine Flexibilität, oder Veränderlichkeit.    &  Die Architektur muss für die Besatzungsmitglieder ihre Fähigkeiten speichern, und, in welchen Sektionen sich diese auswirken. Ebenfalls muss automatisch geprüft werden, ob ein Besatzunsmitglied in der Sektion, in der es sich aktuell befindet, eine Auswirkung hat. 
\\ \hline
\multicolumn{3}{|l|}{{P4.7: Verbesserung Fähigkeiten}} 
\\ \hline
Die Fähigkeiten der Besatzungsmitglieder sollen verbessert werden können. & Keine Flexibilität, oder Veränderlichkeit.    & Die Architektur muss vorsehen, dass die gespeicherten Fähigkeiten nicht konstant sind. Es muss eine Möglichkeit geben für den Spieler, Verbesserungen auszuwählen (z.B. zu kaufen). 
\\ \hline
%
\multicolumn{3}{|l|}{{\textbf{P5: Spielfeld}}} 
\\ \hline         
\multicolumn{3}{|l|}{{P5.1: Spielfeld}} 
\\ \hline
Es soll ein Spielfeld geben, was eine Karte darstellt. Die Karte verzeichent Stationen/Planeten. & Keine Flexibilität, oder Veränderlichkeit.    & Die Architektur muss die möglichen Orte sowie ihre Verbindungen effizient speichern. Es muss ebenfalls immer der relevante Ausschnitt der Karte angezeigt werden. 
\\ \hline          
\multicolumn{3}{|l|}{{P5.2: Reise}} 
\\ \hline
Pro Spielrunde kann eine Station/ein Planet laut den Verbindungen auf der Karte angeflogen werden.  & Keine Flexibilität, oder Veränderlichkeit.    & Die Architektur muss vorsehen, dass der Spieler Eingaben machen kann, zu welchem Gebiet geflogen werden soll. Ebenfalls müssen diese Eingaben überprüft werden (geht dieser Zug mit den existierenden Verbindungen). 
\\ \hline
\multicolumn{3}{|l|}{{P5.3: Ereignisse}} 
\\ \hline
Auf jeder Station/auf jedem Planet soll ein Ereignis auftreten können, was positive oder negative Auswirkungen haben kann; zum Beispiel der Angriff eines feindlichen Schiffes.  & Keine Flexibilität, oder Veränderlichkeit.    &  Die Architektur muss vorsehen, dass mögliche Ereignisse für jeden Ort gespeichert und mit einer gewissen Wahrscheinlichkeit ausgelöst werden. 
\\ \hline
\multicolumn{3}{|l|}{{P5.4: Mehrere Ereignisse}} 
\\ \hline
Es soll mindestens fünf unterschiedliche Arten von Ereignissen geben. & Keine Flexibilität, oder Veränderlichkeit.    & Das Spiel muss möglichst unkompliziert unterschiedliche Ereignisse auslösen und durchführen können. 
\\ \hline
%
\multicolumn{3}{|l|}{{\textbf{P6: Kämpfe}}} 
\\ \hline
\multicolumn{3}{|l|}{{P6.1: Kämpfe}} 
\\ \hline
Es sollen Kämpfe zwischen dem Spieler und feindlichen Raumschiffen möglich sind. & Keine Flexibilität, oder Veränderlichkeit.    & Die Architektur muss vorsehen, dass es abgesehen von Flugrunden auch Kampfrunden gibt, in denen es andere Aktionen gibt.
\\ \hline
\multicolumn{3}{|l|}{{P6.2: Kampfart}} 
\\ \hline
Die Kämpfe sollen in Form von rundenbasierten Entscheidungen gefochten werden. & Keine Flexibilität, oder Veränderlichkeit.    &  Es muss vorgesehen sein, dass der Spieler und der Gegner sich abwechseln, und der Spieler nicht Aktionen durchführen kann, wenn der Gegner dran ist. 
\\ \hline
\multicolumn{3}{|l|}{{P6.3: Kampfhandlungen}} 
\\ \hline
Es sollen folgende Kampfhandlungen möglich sein: Abfeuern einer Waffe auf das gegnerische Schiff, Verteilung der Besatzung auf die Sektionen, und Zuweisung von Energie an Sektionen. & Keine Flexibilität, oder Veränderlichkeit.    &  In den Kampfrunden dürfen nur diese Handlungen zur Verfügung stehen.
\\ \hline
\multicolumn{3}{|l|}{{P6.4: Zeit}} 
\\ \hline
Kampfhandlungen brauchen Zeit: Waffen müssen laden und Besatzungsmitglieder sind nicht sofort in der Zielsektion. & Keine Flexibilität, oder Veränderlichkeit.    & Während den Kampfrunden muss beachtet werden, dass nicht alle Handlungen immer zur Verfügung stehen. 
\\ \hline
%
\multicolumn{3}{|l|}{{\textbf{P7: Spielverlauf}}} 
\\ \hline
\multicolumn{3}{|l|}{{P7.1: Spielverlauf}} 
\\ \hline
Es soll einen klaren Spielverlauf mit einer Endgegner*in un der Möglichkeit, durch Besiegung dieser zu gewinnen, geben.  & Keine Flexibilität, oder Veränderlichkeit.    & Es muss eine übergeordnete Entität geben, die diesen Spielverlauf über Planeten hinweg kontrolliert. 
\\ \hline
\multicolumn{3}{|l|}{{P7.2: Schwierigkeit}} 
\\ \hline
Es sollen unterschiedliche Schwierigkeitsstufen für ein Spiel auswählbar sein. & Keine Flexibilität, oder Veränderlichkeit.    & Es muss möglichst unkomplex Veränderungen geben, die das Spiel leichter/schwerer machen. 
\\ \hline
%
\multicolumn{3}{|l|}{{\textbf{P8: Server}}} 
\\ \hline
\multicolumn{3}{|l|}{{P8.1: Direkter Kampf zweier Spieler}} 
\\ \hline
Es sollen, wenn möglich, zwei Spieler gegeneinander spielen. Diese beiden sollen auch in direkten Kämpfen gegeneinander antreten können. & Keine Flexibilität, oder Veränderlichkeit.  & In der Multiplayer Spielführung müssen die Spieler früher oder später zu Interaktionen gezwungen sein. 
\\ \hline
\multicolumn{3}{|l|}{{P8.2: Computergegner}} 
\\ \hline
Wenn kein zweiter Spieler online ist, soll der Gegner durch den Computer simuliert werden. & Keine Flexibilität, oder Veränderlichkeit.    & Die Architektur muss Computergegner vorsehen, die gegen den Spieler spielen können. 
\\ \hline
\multicolumn{3}{|l|}{{P8.3: Prüfung vom Server}} 
\\ \hline
Der Server soll die Einhaltung von Regeln und Plausibilität während dem Spiel prüfen.  & Keine Flexibilität, oder Veränderlichkeit.    & Im Server muss es etwas geben, was alle Züge prüft. Es ist wichtig, dass keine Züge ausgeführt werden, ohne dass diese verifiziert wurden. 
\\ \hline
\multicolumn{3}{|l|}{{P8.4: Speicherung Spielverlauf}} 
\\ \hline
Der Server soll den Spielverlauf speichern. & Keine Flexibilität, oder Veränderlichkeit.    & Es muss nach jeder Änderung in irgendeiner Form alles so gespeichert werden, dass der Spielverlauf nachvollziehbar ist. 
\\ \hline
\multicolumn{3}{|l|}{{P8.5: Spielverlauf unterbrechen und fortsetzen}} 
\\ \hline
Es soll möglich sein, den Spielverlauf zu unterbrechen und zu einem späteren Zeitpunkt fortzuführen. & Keine Flexibilität, oder Veränderlichkeit.    & Es muss eine Möglichkeit geben, speichern auszuwählen, sowie, bei Spielbeginn ein existierendes Spiel auszuwählen und fortzusetzen. 
\\ \hline
\end{longtable}

\subsection{Probleme und Strategien} \label{sec:strategien}

%{\itshape Aus einer Menge von Faktoren ergeben sich Probleme, die nun in Form 
%von Problemkarten beschrieben werden. Diese resultieren z.\,B.\ aus
%\begin{itemize}
 % \item Grenzen oder Einschränkungen durch Faktoren
 % \item der Notwendigkeit, die Auswirkung eines Faktors zu begrenzen
 % \item der Schwierigkeit, einen Produktfaktor zu erfüllen, oder
 % \item der Notwendigkeit einer allgemeinen Lösung zu globalen Anforderungen.
%\end{itemize}
%Dazu entwickelt Ihr Strategien, um mit den identifizierten Problemen umzugehen.

%Achtet auch hier darauf, dass die Probleme und Strategien wirklich die 
%Architektur betreffen und nicht etwa das Projektmanagement. Die Strategien 
%stellen im Prinzip die Designentscheidungen dar. Sie sollten also die Erklärung 
%für den konkreten Aufbau der verschiedenen Sichten liefern.

%Beschreibt möglichst mehrere Alternativen und gebt an, für welche Ihr Euch 
%letztlich aus welchem Grunde entschieden habt. Natürlich müssen die genannten 
%Strategien in den folgenden Sichten auch tatsächlich umgesetzt werden!

%Ein sehr häufiger Fehler ist es, dass SWP-Gruppen arbeitsteilig vorgehen: die 
%eine Gruppe schreibt das Kapitel zur Analyse von Faktoren und zu den Strategien, 
%die andere Gruppe beschreibt die diversen Sichten, ohne dass diese beiden 
%Gruppen sich abstimmen. Natürlich besteht aber ein Zusammenhang zwischen den
%Faktoren, Strategien und Sichten. Dieser muss erkennbar sein, indem sich die 
%verschiedenen Kapitel eindeutig aufeinander beziehen.}

\begin{table}[H]
    \centering
    \begin{tabular}{|p{15cm}|}
    \hline
          \textbf{Problem 1:} Begrenzte Zeit \\ \hline
         Die Zeit, die wir für die Implementierung des Projekts haben, ist sehr begrenzt, was dazu führen kann, dass wir Zeitprobleme bekommen und eventuell nicht alle Aspekte implementieren können. \\ \hline
          \textbf{Einflussfaktoren: } \\
		O1.2 \\
		O1.4 \\
		O2 \\
		O3 \\
          \hline
          \textbf{Strategien} \\ \hline
            {\phantomsection}          
           \label{strategie:1.1}     
          \textbf{Strategie 1.1: Modularisierung} Die Aufgaben werden modularisiert, sodass kleine Aufgaben für kurze Zeiträume geschaffen werden. \\        
  {\phantomsection}          
           \label{strategie:1.2}              
          \textbf{Strategie 1.2: Inkrementelle Entwicklung} Da eventuell nicht alle Aspekte implementiert werden können, werden zunächst die Mindestanforderungen implementiert, die für das Bestehen zwingend notwendig sind. Danch werden eventuelle zusätzliche Funktionen beachtet. \\
	 {\phantomsection}          
           \label{strategie:1.3}     
          \textbf{Strategie 1.3: Austausch} Es wird sich regelmäßig über Fortschritte und Probleme ausgetauscht. \\ 
	 \\ \hline
    \end{tabular}

    \caption{Zeit Problemkarte}
    \label{tab:ProblemKarte1}
\end{table}
Hier werden wir alle drei Strategien anwenden, um eine möglichst erfolgreiche Implementierung zu erreichen. \\

\begin{table}[H]
    \centering
    \begin{tabular}{|p{15cm}|}
    \hline
          \textbf{Problem 2:} LibGDX \\ \hline
	Die Gruppenmitglieder haben wenig bis keine Erfahrung mit LibGDX, was eine benötigte Anforderung ist. \\
         \\ \hline
          \textbf{Einflussfaktoren: } \\
	T2.1 \\
	T2.2 \\
	O1.2 \\
	O1.3 \\
          \hline
          \textbf{Strategien} \\ \hline
            {\phantomsection}          
           \label{strategie:2.1}     
          \textbf{Strategie 2.1: Anlernen} Die Gruppenmitglieder müssen sich so früh wie möglich mit der Verwendung von LibGDX vertraut machen, damit dies idealerweise auch schon in die Architektur mit einfließen kann.  \\        
	 \\ \hline
    \end{tabular}

    \caption{LibGDX Problemkarte}
    \label{tab:ProblemKarte2}
\end{table}
Hier werden wir die erste Strategie verwenden. \\

\begin{table}[H]
    \centering
    \begin{tabular}{|p{15cm}|}
    \hline
          \textbf{Problem 3: Spielstand}  \\ \hline
	Der Spielstand soll gespeichert werden, und zu späteren Zeitpunkten wieder aufgerufen. Dazu benötigt der Server eine Datenbank. \\
         \\ \hline
          \textbf{Einflussfaktoren: } \\
	O1.2 \\
	T1.3 \\
	T1.2 \\
	T1.1 \\
          \hline
          \textbf{Strategien} \\ \hline
            {\phantomsection}          
           \label{strategie:3.1}     
          \textbf{Strategie 3.1: H2}  Wir verwenden die Datenbank H2. \\        
  {\phantomsection}          
           \label{strategie:3.2}              
          \textbf{Strategie 3.2: SQLite} Wir verwenden die Datenbank SQLite. \\
	 {\phantomsection}          
           \label{strategie:3.3}     
          \textbf{Strategie 3.3: Derby } Wir verwenden die Datenbank Derby. \\ 
	 \\ \hline
    \end{tabular}

    \caption{Spielstand Problemkarte}
    \label{tab:ProblemKarte3}
\end{table}
Wir setzen Strategie .. um. \\

\begin{table}[H]
    \centering
    \begin{tabular}{|p{15cm}|}
    \hline
          \textbf{Problem 4: Raumschiff Größen}  \\ \hline
	Raumschiffe sollen verschiedene Größen haben. Es soll mindestens drei Raumschiffe mit individuellem Sektionenlayout geben. \\
         \\ \hline
          \textbf{Einflussfaktoren: } \\
	P1.1 \\
	P1.2 \\
	P3.1 \\
	P3.4 \\
          \hline
          \textbf{Strategien} \\ \hline
            {\phantomsection}          
           \label{strategie:4.1}     
          \textbf{Strategie 4.1: Feste Raumschiffe} Wir erstellen feste Raumschiffe, aus denen der Spieler eines auswählt. \\        
  {\phantomsection}          
           \label{strategie:4.2}              
          \textbf{Strategie 4.2: Raumschiff Selektor} Wir haben einen Raumschiff Editor, in dem der Spieler sich am Anfang des Spieles ein Raumschiff zusammenbauen kann. \\
	 \\ \hline
    \end{tabular}

    \caption{Raumschiffgrößen Problemkarte}
    \label{tab:ProblemKarte4}
\end{table}
Hier setzen wir die erste Strategie um. \\

\begin{table}[H]
    \centering
    \begin{tabular}{|p{15cm}|}
    \hline
          \textbf{Problem 5: Multiplayer}  \\ \hline
	Die Spieler müssen vom Server gegeneinander in Kämpfen antreten gelassen werden. \\
         \\ \hline
          \textbf{Einflussfaktoren: } \\
	T1.6 \\
	T1.3 \\
	T1.2 \\
	P8.1 \\
          \hline
          \textbf{Strategien} \\ \hline
            {\phantomsection}          
           \label{strategie:5.1}     
          \textbf{Strategie 5.1: Bestimmte Anzahl}  Der Server lässt die Spieler nach einer bestimmten Anzahl Runden gegeneinander antreten. \\        
  {\phantomsection}          
           \label{strategie:5.2}              
          \textbf{Strategie 5.2: Zufällige Anzahl} Der Server lässt die Spieler nach einer zufälligen Anzahl Runden gegeneinander antreten.   \\
	 {\phantomsection}          
           \label{strategie:5.3}     
          \textbf{Strategie 5.3: Zwangsläufig} Das Spiel ist so konzipiert, dass die Spieler zwangsläufig gegeneinander antreten müssen (z.B. über Quests: "verteidige den Planeten xy" und "Erobere den Planeten xy", Einzigartige Ressourcen, die für bestimmte Aufgaben/Upgrades oÄ erforderlich sind aber nur einmal im Spiel vorkommen und deshalb irgendwann nur noch von einem Mitspieler erbeutet werden können.  \\ 
	 \\ \hline
    \end{tabular}

    \caption{Multiplayer Problemkarte}
    \label{tab:ProblemKarte5}
\end{table}
Hier setzen wir die .. Strategie um. \\

\begin{table}[H]
    \centering
    \begin{tabular}{|p{15cm}|}
    \hline
          \textbf{Problem 6: Neue Gruppenmitglieder}  \\ \hline
	Da nur vier der Gruppenmitglieder bereits vorher miteinander gearbeitet haben, müssen wir uns als Gruppe neu zusammen einarbeiten, und feststellen, wie unsere Zusammenarbeit am besten funktionieren kann.\\
         \\ \hline
          \textbf{Einflussfaktoren: } \\
	O1.1 \\
	O1.2 \\
	O1.3 \\
	O1.4 \\
          \hline
          \textbf{Strategien} \\ \hline
            {\phantomsection}          
           \label{strategie:6.1}     
          \textbf{Strategie 6.1: Vorstellungsrunde} Wir machen eine Vorstellungsrunde, wo wir unsere Stärken und Schwächen besprechen, und unsere bisherigen Erfahrungen mit Gruppenarbeit besprechen. \\        
  {\phantomsection}          
           \label{strategie:6.2}              
          \textbf{Strategie 6.2: Reflektionen} Wir machen regelmäßige Reflektionen, um festzustellen, was funktioniert und was nicht funktioniert. \\
	 \\ \hline
    \end{tabular}

    \caption{Gruppenmitglieder Problemkarte}
    \label{tab:ProblemKarte6}
\end{table}
Hier setzen wir beide Strategien um. \\

\begin{table}[H]
    \centering
    \begin{tabular}{|p{15cm}|}
    \hline
          \textbf{Problem 7: Zugüberprüfung}  \\ \hline
	Jeder Zug wird vom Server auf Korrektheit geprüft, bevor er an den anderen Spieler übermittelt wird. \\
         \\ \hline
          \textbf{Einflussfaktoren: } \\
	T1.6 \\
	T1.5 \\
	P8.3 \\
	P8.4 \\
          \hline
          \textbf{Strategien} \\ \hline
            {\phantomsection}          
           \label{strategie:7.1}     
          \textbf{Strategie 7.1: }  \\        
  {\phantomsection}          
           \label{strategie:7.2}              
          \textbf{Strategie 7.2:}  \\
	 {\phantomsection}          
           \label{strategie:7.3}     
          \textbf{Strategie 7.3: }  \\ 
	 \\ \hline
    \end{tabular}

    \caption{Zugüberprüfung Problemkarte}
    \label{tab:ProblemKarte7}
\end{table}
Hier setzen wir die .. Strategie um. \\

\begin{table}[H]
    \centering
    \begin{tabular}{|p{15cm}|}
    \hline
          \textbf{Problem 8: Java}  \\ \hline
	Es soll eine Java Applikation sein, es soll mindestens java 8 sein (oder höher). \\
         \\ \hline
          \textbf{Einflussfaktoren: } \\
	T1.1 \\
	T1.6 \\
	T1.3 \\
          \hline
          \textbf{Strategien} \\ \hline
            {\phantomsection}          
           \label{strategie:8.1}     
          \textbf{Strategie 8.1:}  \\        
  {\phantomsection}          
           \label{strategie:8.2}              
          \textbf{Strategie 8.2:}  \\
	 {\phantomsection}          
           \label{strategie:8.3}     
          \textbf{Strategie 8.3: }  \\ 
	 \\ \hline
    \end{tabular}

    \caption{Java Problemkarte}
    \label{tab:ProblemKarte8}
\end{table}
Hier setzen wir die .. Strategie um. \\

\begin{table}[H]
    \centering
    \begin{tabular}{|p{15cm}|}
    \hline
          \textbf{Problem 9: Reise}  \\ \hline
	Es sollen Reisen von Stern zu Stern möglich sein. Dafür muss es eine Karte geben, auf der mehrere Planeten/Sternen/Stationen geben kann, über die man sich bewegen kann. \\
         \\ \hline
          \textbf{Einflussfaktoren: } \\
	T1.6 \\
	P5.1 \\
	P5.2 \\
          \hline
          \textbf{Strategien} \\ \hline
            {\phantomsection}          
           \label{strategie:9.1}     
          \textbf{Strategie 9.1: Listen} Jede Karte ist eine Liste, in der jeder Eintrag ein/e Planet/Stern/Station ist. Die Verbindungen werden extra gespeichert.  \\        
  {\phantomsection}          
           \label{strategie:9.2}              
          \textbf{Strategie 9.2: TiledMaps} Wir benutzen TiledMaps von libGDX.  \\
	 \\ \hline
    \end{tabular}

    \caption{Reise Problemkarte}
    \label{tab:ProblemKarte9}
\end{table}
Hier setzen wir die .. Strategie um. \\

\begin{table}[H]
    \centering
    \begin{tabular}{|p{15cm}|}
    \hline
          \textbf{Problem 10: Waffen}  \\ \hline
	Es muss eine Auswahl von Waffen mit verschiedenen Schadenshöhen, Cooldown, Trefferwahrscheinlichkeit und weiteren Effekten (wie z.B. Einfrieren, Personenschaden in den Sektionen) geben. \\
         \\ \hline
          \textbf{Einflussfaktoren: } \\
	P3.2 \\
	P3.3 \\
          \hline
          \textbf{Strategien} \\ \hline
            {\phantomsection}          
           \label{strategie:10.1}     
          \textbf{Strategie 10.1:}  \\        
  {\phantomsection}          
           \label{strategie:10.2}              
          \textbf{Strategie 10.2:}  \\
	 {\phantomsection}          
           \label{strategie:10.3}     
          \textbf{Strategie 10.3: }  \\ 
	 \\ \hline
    \end{tabular}

    \caption{Waffen Problemkarte}
    \label{tab:ProblemKarte10}
\end{table}
Hier setzen wir die .. Strategie um. \\

\begin{table}[H]
    \centering
    \begin{tabular}{|p{15cm}|}
    \hline
          \textbf{Problem 11: Cross Platform Support}  \\ \hline
	Das Spiel soll auf verschiedenen Plattformen laufen. Für die Implementierung muss die Performance beachtet werden. \\
         \\ \hline
          \textbf{Einflussfaktoren: } \\
	T1.2 \\
          \hline
          \textbf{Strategien} \\ \hline
            {\phantomsection}          
           \label{strategie:11.1}     
          \textbf{Strategie 11.1:}  \\        
  {\phantomsection}          
           \label{strategie:11.2}              
          \textbf{Strategie 11.2:}  \\
	 {\phantomsection}          
           \label{strategie:11.3}     
          \textbf{Strategie 11.3: }  \\ 
	 \\ \hline
    \end{tabular}

    \caption{Cross Platform Problemkarte}
    \label{tab:ProblemKarte11}
\end{table}
Hier setzen wir die .. Strategie um. \\

\begin{table}[H]
    \centering
    \begin{tabular}{|p{15cm}|}
    \hline
          \textbf{Problem 12: Keine persönlichen Treffen}  \\ \hline
	Auf Grund der globalen COVID-19 Pandemie wird es nicht möglich sein, uns persönlich zu treffen. Wir werden die komplette Implementierungsphase auf digitalem Weg kompensieren müssen. \\
         \\ \hline
          \textbf{Einflussfaktoren: } \\
	O1.4 \\
          \hline
          \textbf{Strategien} \\ \hline
            {\phantomsection}          
           \label{strategie:12.1}     
          \textbf{Strategie 12.1: Discord} Teammeetings werden über Discord abgehalten.  \\        
  {\phantomsection}          
           \label{strategie:12.2}              
          \textbf{Strategie 12.2: Aufgabenlisten} Wir erstellen digitale Aufgabenlisten, um einen Überblick über die Aufgaben, die es gibt, zu haben.  \\
	 {\phantomsection}          
           \label{strategie:12.3}     
          \textbf{Strategie 12.3: Merge-partys} Um Fehler beim mergen unserer git Branches zu vermeiden, mergen wir nur, wenn mindestens die Häflte der Teammitglieder anwesend ist, und verifizieren kann, dass z.B. Konflikte korrekt gelöst werden. \\ 
	 \\ \hline
    \end{tabular}

    \caption{Persönliche Treffen Problemkarte}
    \label{tab:ProblemKarte12}
\end{table}
Hier wenden wir alle drei Strategien an. \\

\begin{table}[H]
    \centering
    \begin{tabular}{|p{15cm}|}
    \hline
          \textbf{Problem 13: Kampfrundeninterne Speicherung}  \\ \hline
	Es muss möglich sein, innerhalb eines Kampfes das Spiel zu unterbrechen und später unter den selben Bedingungen fortzusetzen. \\
         \\ \hline
          \textbf{Einflussfaktoren: } \\
	P8.4 \\
	P8.5 \\
	T1.6 \\
	T1.3 \\
          \hline
          \textbf{Strategien} \\ \hline
            {\phantomsection}          
           \label{strategie:13.1}     
          \textbf{Strategie 13.1: gesamter Spielstand} Der gesamte Spielstand wird bei jeder Änderung gespeichert.  \\        
  {\phantomsection}          
           \label{strategie:13.2}              
          \textbf{Strategie 13.2: Änderungen} Es wir der initiale Spielstand und danach nur alle Änderungen gegenüber der letzen Speicherung gespeichert.  \\
	 \\ \hline
    \end{tabular}

    \caption{Kampfrundeninterne Speicherung Problemkarte}
    \label{tab:ProblemKarte13}
\end{table}
Hier werden wir die .. Strategie umsetzen. \\

\begin{table}[H]
    \centering
    \begin{tabular}{|p{15cm}|}
    \hline
          \textbf{Problem 14: Endgegener}  \\ \hline
	Es muss einen Endgegner geben, der zum Sieg zwingend geschlagen werden muss. Entweder muss sichergestellt werden, dass beim Multiplayer ein Kampf gegen den zweiten Spieler am "Ende" des Spieles steht oder es muss mindestens einen NPC geben. \\
         \\ \hline
          \textbf{Einflussfaktoren: } \\
	P7.1 \\
	P8.1 \\
	P8.2 \\
	T1.6 \\
	T1.5 \\
          \hline
          \textbf{Strategien} \\ \hline
            {\phantomsection}          
           \label{strategie:14.1}     
          \textbf{Strategie 14.1: NPC} Es gibt einen NPC, der besiegt werden muss, um zu gewinnen.  \\        
  {\phantomsection}          
           \label{strategie:14.2}              
          \textbf{Strategie 14.2: zweiter Spieler} Der zweite Spieler ist der Endgegner, und das Spiel endet nach einem Kampf.  \\
	 {\phantomsection}          
           \label{strategie:14.3}     
          \textbf{Strategie 14.3: Zeitpunkt} Es gibt einen Zeitpunkt im Spiel, nach dem der nächste Kampf zwischen den beiden Spielern zum "Endgegner" wird.  \\ 
	 \\ \hline
    \end{tabular}

    \caption{Endgegner Problemkarte}
    \label{tab:ProblemKarte14}
\end{table}
Hier setzen wird die Strategie .. um. \\

\begin{table}[H]
    \centering
    \begin{tabular}{|p{15cm}|}
    \hline
          \textbf{Problem 15: Unterschiedliche Besatzungsmitglieder}  \\ \hline
	Die Besatzungsmitglieder müssen unterschiedliche Eigenschaften (Attribute) haben, die sich auf das Raumschiff auswirken und im Laufe des Spieles verbessert werden können. \\
         \\ \hline
          \textbf{Einflussfaktoren: } \\
	P4 \\
	T1.6 \\
	T1.5 \\
	P6.3 \\
	P6.4 \\
	P8.2 \\
	P8.4 \\
          \hline
          \textbf{Strategien} \\ \hline
            {\phantomsection}          
           \label{strategie:15.1}     
          \textbf{Strategie 15.1: }  \\        
  {\phantomsection}          
           \label{strategie:15.2}              
          \textbf{Strategie 15.2:}  \\
	 {\phantomsection}          
           \label{strategie:15.3}     
          \textbf{Strategie 15.3: }  \\ 
	 \\ \hline
    \end{tabular}

    \caption{Unterschiedliche Besatzungsmitglieder Problemkarte}
    \label{tab:ProblemKarte15}
\end{table}
Hier setzen wir die .. Strategie um.  \\

\begin{table}[H]
    \centering
    \begin{tabular}{|p{15cm}|}
    \hline
          \textbf{Problem 16: Prozess Overhead}  \\ \hline
	Aufgrund der Ermängelung an Möglichkeiten zum räumlichen gemeinschaftlichen Arbeiten und der Beschränkung des persönlichen Kontaktes auf "Telefonate" zur Besprechung besteht das Risiko, dass im Zuge dieser Meetings ausschweifende und irgendwann nicht mehr lösungsorientierte Diskussionen entstehen. \\
         \\ \hline
          \textbf{Einflussfaktoren: } \\
	O1.1 \\
	O1.4 \\
	OCorona \\
          \hline
          \textbf{Strategien} \\ \hline
            {\phantomsection}          
           \label{strategie:16.1}     
          \textbf{Strategie 16.1: Begrenzte Zeitfenster} Für die Meetings setzen wir begrenzte Zeitfenster fest.  \\        
  {\phantomsection}          
           \label{strategie:16.2}              
          \textbf{Strategie 16.2: Themen} Vor den Meetings setzen wir Themen fest, die besprochen werden sollen.  \\
	 \\ \hline
    \end{tabular}

    \caption{Prozess Overhead Problemkarte}
    \label{tab:ProblemKarte16}
\end{table}
Hier wenden wir beide Strategien an.  \\

\begin{table}[H]
    \centering
    \begin{tabular}{|p{15cm}|}
    \hline
          \textbf{Problem 17: Schwierigkeitsstufen}  \\ \hline
	Das Spiel soll unterschiedliche Schwierigkeitsstufen haben, aus denen der Spieler am Anfang auswählen kann. \\
         \\ \hline
          \textbf{Einflussfaktoren: } \\
	P7.2 \\
	P8.2 \\
	P7.1 \\
          \hline
          \textbf{Strategien} \\ \hline
            {\phantomsection}          
           \label{strategie:17.1}     
          \textbf{Strategie 17.1: Attribute} Bestimmte Attribute werden am Anfang des Spieles abhängig von der Schwierigkeitsstufe gestellt (z.B. Trefferwahrscheinlichkeit, Schadenshöhe, ...)  \\        
  {\phantomsection}          
           \label{strategie:17.2}              
          \textbf{Strategie 17.2: Unterschiedliche Computergegner} Es gibt unterschiedlich schwierige Computergegner für die unterschiedlichen Schwierigkeitsstufen.   \\
	 {\phantomsection}          
           \label{strategie:17.3}     
          \textbf{Strategie 17.3: Karten } Es gibt Unterschiede in den Karten (z.B. Anzahl der schlechten Ereignisse, mehr/weniger Verbindungen zwischen Karten sodass man mehr reisen muss, ...)  \\ 
	 \\ \hline
    \end{tabular}

    \caption{Schwierigkeitsstufen Problemkarte}
    \label{tab:ProblemKarte17}
\end{table}
Hier werden wir die ... Strategie umsetzen. \\

\begin{table}[H]
    \centering
    \begin{tabular}{|p{15cm}|}
    \hline
          \textbf{Problem 18: Rundenbasiert}  \\ \hline
	Da das Spiel rundenbasiert ist, kann es sein, dass ein Spieler (vor allem im Zweispieler) sehr lange braucht, um einen Zug zu machen. \\
         \\ \hline
          \textbf{Einflussfaktoren: } \\
	P6.2 \\
	P2.1 \\
	P2.3 \\
	T1.3 \\
	T1.5 \\
	T1.6 \\
          \hline
          \textbf{Strategien} \\ \hline
            {\phantomsection}          
           \label{strategie:18.1}     
          \textbf{Strategie 18.1: Zeitbegrenzung} Es gitb eine Zeitbegrenzung, sodass der Spielfluss erhalten bleibt.   \\        
  {\phantomsection}          
           \label{strategie:18.2}              
          \textbf{Strategie 18.2: Keine Zeitbegrenzung} Es gibt keine Zeitbegrenzung. \\
	 \\ \hline
    \end{tabular}

    \caption{Rundenbasiert Problemkarte}
    \label{tab:ProblemKarte18}
\end{table}
Hier setzen wir die ... Strategie um. \\

\begin{table}[H]
    \centering
    \begin{tabular}{|p{15cm}|}
    \hline
          \textbf{Problem 19: Zerstörung von wichtigen Systemen}  \\ \hline
	Durch gegnerische Angriffe können wichtige Systeme (z.B. Antrieb) zerstört werden. Wenn die Besatzung tot ist, kann das nicht mehr repariert werden, und das Schiff ist nutzlos. \\
         \\ \hline
          \textbf{Einflussfaktoren: } \\
	P1.3 \\
	P1.4 \\
	P4.3 \\
	P4.4 \\
	T1.3 \\
	T1.5 \\
	T1.6 \\
          \hline
          \textbf{Strategien} \\ \hline
            {\phantomsection}          
           \label{strategie:19.1}     
          \textbf{Strategie 19.1: Testen} Es muss getestet werden, ob der Spieler noch Züge machen kann.  \\        
  {\phantomsection}          
           \label{strategie:19.2}              
          \textbf{Strategie 19.2: Knopf} Es gibt einen Aufgeben Knopf, den der Spieler jeder Zeit drücken kann, um das Spiel zu beenden.  \\
	 \\ \hline
    \end{tabular}

    \caption{Zerstörung von wichtigen Systemen Problemkarte}
    \label{tab:ProblemKarte19}
\end{table}
Hier setzen wir die .. Strategie um. \\

\begin{table}[H]
    \centering
    \begin{tabular}{|p{15cm}|}
    \hline
          \textbf{Problem 20: Schutzschilde und Hüllen }  \\ \hline
	Schutzschilde und Außenhülle müssen einen Einfluss auf die Treffer haben (es soll also unabhängig von der Sektion auf die geziehlt wird erst der Schutz zerstört werden / von dessem Status abgezogen werden). \\
         \\ \hline
          \textbf{Einflussfaktoren: } \\
	P1.3 \\
	P1.4 \\
	P2.1 \\
	P3.1 \\
	P2.2 \\
	P2.3 \\
	P6.3 \\
	P3.4 \\
	T1.3 \\
	T1.5 \\
	T1.6 \\
          \hline
          \textbf{Strategien} \\ \hline
            {\phantomsection}          
           \label{strategie:20.1}     
          \textbf{Strategie 20.1:}  \\        
  {\phantomsection}          
           \label{strategie:20.2}              
          \textbf{Strategie 20.2:}  \\
	 {\phantomsection}          
           \label{strategie:20.3}     
          \textbf{Strategie 20.3: }  \\ 
	 \\ \hline
    \end{tabular}

    \caption{Schutzschilde und Hüllen Problemkarte}
    \label{tab:ProblemKarte20}
\end{table}
Hier setzen wir die .. Strategie um. \\

\begin{table}[H]
    \centering
    \begin{tabular}{|p{15cm}|}
    \hline
          \textbf{Problem 21: Treffen von Sektionen}  \\ \hline
	Wenn eine Sektion getroffen wird, müssen alle zugehörigen Dinge (Besatzung, System) mit betroffen werden. \\
         \\ \hline
          \textbf{Einflussfaktoren: } \\
	P1.3\\
	P1.4 \\
	P2.2 \\
	P2.3 \\
	P3.2 \\
	P3.3 \\
	P4.4 \\
	P6.3 \\
	T1.3 \\
	T1.5 \\
	T1.6 \\
          \hline
          \textbf{Strategien} \\ \hline
            {\phantomsection}          
           \label{strategie:21.1}     
          \textbf{Strategie 21.1:}  \\        
  {\phantomsection}          
           \label{strategie:21.2}              
          \textbf{Strategie 21.2:}  \\
	 {\phantomsection}          
           \label{strategie:21.3}     
          \textbf{Strategie 21.3: }  \\ 
	 \\ \hline
    \end{tabular}

    \caption{Treffen von Sektionen Problemkarte}
    \label{tab:ProblemKarte21}
\end{table}
Hier setzen wir die .. Strategie um. \\

\begin{table}[H]
    \centering
    \begin{tabular}{|p{15cm}|}
    \hline
          \textbf{Problem 22: Negative Vorerfahrungen}  \\ \hline
	Da die meisten Gruppenmitglieder einen schlechten SWP-Verlauf hinter sich haben, besteht die Gefahr, dass für die verantwortlichen Probleme überkompensiert wird. \\
         \\ \hline
          \textbf{Einflussfaktoren: } \\
	O1.4 \\
          \hline
          \textbf{Strategien} \\ \hline
            {\phantomsection}          
           \label{strategie:22.1}     
          \textbf{Strategie 22.1: } Wir müssen aus den Fehlern vom letzten Mal lernen, ohne andere Dinge aus den Augen zu verlieren oder uns mit Vorsorgemaßnahmen zu verhindern.  \\     
	 \\ \hline
    \end{tabular}

    \caption{Negative Vorerfahrungen Problemkarte}
    \label{tab:ProblemKarte22}
\end{table}
Hier wenden wir die erste Strategie an. \\

\begin{table}[H]
    \centering
    \begin{tabular}{|p{15cm}|}
    \hline
          \textbf{Problem 23: Schiffe vom Computer spielbar}  \\ \hline
	Wir brauchen einen NPC für den Multiplayer, der das Spiel ausreichend beherrscht. \\
         \\ \hline
          \textbf{Einflussfaktoren: } \\
	T1.6 \\
	P6.3 \\
	P7.2 \\
	P8.1 \\
	P8.2 \\
	T1.3 \\
          \hline
          \textbf{Strategien} \\ \hline
            {\phantomsection}          
           \label{strategie:23.1}     
          \textbf{Strategie 23.1:}  \\        
  {\phantomsection}          
           \label{strategie:23.2}              
          \textbf{Strategie 23.2:}  \\
	 {\phantomsection}          
           \label{strategie:23.3}     
          \textbf{Strategie 23.3: }  \\ 
	 \\ \hline
    \end{tabular}

    \caption{NPC Problemkarte}
    \label{tab:ProblemKarte23}
\end{table}
Hier setzen wir die .. Strategie um. \\

%TODO: PROBLEMKARTE ENDBOSS DOPPELT?

\begin{table}[H]
    \centering
    \begin{tabular}{|p{15cm}|}
    \hline
          \textbf{Problem 24: Balancing}  \\ \hline
	Da es Ansprüche an die Schwierigkeit gibt, müssen die Waffen von der Stärke her verleichbar sein. \\
         \\ \hline
          \textbf{Einflussfaktoren: } \\
	P1 \\
	P2 \\
	P3 \\
	P4.3 - P4.7 \\
	P5.3 \\
	P6.2 \\
	P6.3 \\
	P7.2 \\
	P8.1 \\
	P8.2 \\
          \hline
          \textbf{Strategien} \\ \hline
            {\phantomsection}          
           \label{strategie:24.1}     
          \textbf{Strategie 24.1:}  \\        
  {\phantomsection}          
           \label{strategie:24.2}              
          \textbf{Strategie 24.2:}  \\
	 {\phantomsection}          
           \label{strategie:24.3}     
          \textbf{Strategie 24.3: }  \\ 
	 \\ \hline
    \end{tabular}

    \caption{Balancing Problemkarte}
    \label{tab:ProblemKarte24}
\end{table}
Hier setzen wir die .. Strategie um. \\

\begin{table}[H]
    \centering
    \begin{tabular}{|p{15cm}|}
    \hline
          \textbf{Problem 25: Kompetenz der Entwickler}  \\ \hline
	Die unterschiedlichen Entwickler haben unterschiedliche (weitreichende) Kentnisse der Technologien. \\
         \\ \hline
          \textbf{Einflussfaktoren: } \\
	O1.1 \\
	O1.4 \\
	T1 \\
          \hline
          \textbf{Strategien} \\ \hline
            {\phantomsection}          
           \label{strategie:25.1}     
          \textbf{Strategie 25.1: Modularisierung} Wir teilen das Projekt in Module auf, sodass die unterschiedlichen Module möglichst abgekapselt voneinander implementiert werden können. \\        
  {\phantomsection}          
           \label{strategie:25.2}              
          \textbf{Strategie 25.2: Kleingruppen} Wir teilen uns in kleinere Gruppen auf, die jeweils für einen Aspekt "Experten" werden. Dadurch können wir idealerweise alle an dem arbeiten, was uns am meisten passt, und ebenfalls müssen wir uns somit nicht in alles einarbeiten, was wir nicht können, sondern nur Teile. \\
	 {\phantomsection}          
           \label{strategie:25.3}     
          \textbf{Strategie 25.3: }  \\ 
	 \\ \hline
    \end{tabular}

    \caption{Kompetenz der Entwickler Problemkarte}
    \label{tab:ProblemKarte25}
\end{table}
Hier setzen wir beide Strategien durch. \\

\begin{table}[H]
    \centering
    \begin{tabular}{|p{15cm}|}
    \hline
          \textbf{Problem 26: Sektionen}  \\ \hline
	Das Schiff / die Schiffe sollen in systemrelevante Sektionen aufgeteilt sein. \\
         \\ \hline
          \textbf{Einflussfaktoren: } \\
	P1.1 - P1.3 \\
	P2.1 \\
	P3.1 \\
          \hline
          \textbf{Strategien} \\ \hline
            {\phantomsection}          
           \label{strategie:26.1}     
          \textbf{Strategie 26.1:}  \\        
  {\phantomsection}          
           \label{strategie:26.2}              
          \textbf{Strategie 26.2:}  \\
	 {\phantomsection}          
           \label{strategie:26.3}     
          \textbf{Strategie 26.3: }  \\ 
	 \\ \hline
    \end{tabular}

    \caption{Sektionen Problemkarte}
    \label{tab:ProblemKarte26}
\end{table}
Hier setzen wir die .. Strategie um. \\

\begin{table}[H]
    \centering
    \begin{tabular}{|p{15cm}|}
    \hline
          \textbf{Problem 27: Sektionenattribute}  \\ \hline
	Sektionen sollen Attribute haben, die auch geupradet werden können. Sie sollen ebenfalls beschädigt und repariert werden können. \\
         \\ \hline
          \textbf{Einflussfaktoren: } \\
	P1 \\
	P2 \\
	P3.4 \\
	P4.3 \\
	P5.3 \\
	P6.3 \\
	P8.4 \\
	T1.3 \\
	T1.5 \\
	T1.6 \\
          \hline
          \textbf{Strategien} \\ \hline
            {\phantomsection}          
           \label{strategie:27.1}     
          \textbf{Strategie 27.1:}  \\        
  {\phantomsection}          
           \label{strategie:27.2}              
          \textbf{Strategie 27.2:}  \\
	 {\phantomsection}          
           \label{strategie:27.3}     
          \textbf{Strategie 27.3: }  \\ 
	 \\ \hline
    \end{tabular}

    \caption{Sektionenattribute Problemkarte}
    \label{tab:ProblemKarte27}
\end{table}
Hier setzen wir die .. Strategie um.\\

\begin{table}[H]
    \centering
    \begin{tabular}{|p{15cm}|}
    \hline
          \textbf{Problem 28: Kampf- und Flugrunden}  \\ \hline
	Es muss zwischen Runden, in denen ein Kampf gestartet oder ein Planet angeflogen werden kann und Runden innerhalb des Kampfes unterschieden werden. \\
         \\ \hline
          \textbf{Einflussfaktoren: } \\
	P6.4 \\	
	P7.1 \\
	P5.2 \\
	P6.3 \\
	T1.6 \\
	T1.3 \\
          \hline
          \textbf{Strategien} \\ \hline
            {\phantomsection}          
           \label{strategie:28.1}     
          \textbf{Strategie 28.1:}  \\        
  {\phantomsection}          
           \label{strategie:28.2}              
          \textbf{Strategie 28.2:}  \\
	 {\phantomsection}          
           \label{strategie:28.3}     
          \textbf{Strategie 28.3: }  \\ 
	 \\ \hline
    \end{tabular}

    \caption{Kampf- und Flugrunden Problemkarte}
    \label{tab:ProblemKarte28}
\end{table}
Hier setzen wir die .. Strategie um. \\

%%%%%%%%%%%%%%%%%%%%%%%%%%%%%%%%%%%%%%%%%%%%%%%%%%%%%%%%%%%%%%%%%%%%%%%%
\section{Konzeptionelle Sicht} \label{sec:konzeptionell}

{\itshape Diese Sicht beschreibt das System auf einer hohen Abstraktionsebene,
d.\,h. mit sehr starkem Bezug zur Anwendungsdomäne und den geforderten
Produktfunktionen und "~attributen. Sie legt die Grobstruktur fest, ohne gleich 
in die Details von spezifischen Technologien abzugleiten. Sie wird in den 
nachfolgenden Sichten konkretisiert und verfeinert. Die konzeptionelle Sicht 
wird mit {UML}-Komponentendiagrammen visualisiert.}


%%%%%%%%%%%%%%%%%%%%%%%%%%%%%%%%%%%%%%%%%%%%%%%%%%%%%%%%%%%%%%%%%%%%%%%%
\section{Modulsicht} \label{sec:modulsicht}

{\itshape Diese Sicht beschreibt den statischen Aufbau des Systems mit Hilfe von
Modulen, Subsystemen, Schichten und Schnittstellen. Diese Sicht ist 
hierarchisch, d.\,h. Module werden in Teilmodule zerlegt. Die Zerlegung endet 
bei Modulen, die ein klar umrissenes Arbeitspaket für eine Person darstellen und
in einer Kalenderwoche implementiert werden können. Die Modulbeschreibung der 
Blätter dieser Hierarchie muss genau genug und ausreichend sein, um das Modul 
implementieren zu können.

Die Modulsicht wird durch {UML}-Paket- und Klassendiagramme visualisiert.

Die Module werden durch ihre Schnittstellen beschrieben.
Die Schnittstelle eines Moduls $M$ ist die Menge aller Annahmen, die andere 
Module über $M$ machen dürfen, bzw.\ jene Annahmen, die $M$ über seine 
verwendeten Module macht (bzw. seine Umgebung, wozu auch Speicher, Laufzeit 
etc.\ gehören).
Konkrete Implementierungen dieser Schnittstellen sind das Geheimnis des Moduls
und können vom Programmierer festgelegt werden. Sie sollen hier dementsprechend 
nicht beschrieben werden. 

Die Diagramme der Modulsicht sollten die zur Schnittstelle gehörenden Methoden
enthalten. Die Beschreibung der einzelnen Methoden (im Sinne der 
Schnittstellenbeschreibung) geschieht allerdings per Javadoc im zugehörigen 
Quelltext. Das bedeutet, dass Ihr für alle Eure Module Klassen, Interfaces und 
Pakete erstellt und sie mit den Methoden der Schnittstellen verseht. Natürlich 
noch ohne Methodenrümpfe bzw.\ mit minimalen Rümpfen. Dieses Vorgehen 
vereinfacht den Schnittstellenentwurf und stellt Konsistenz sicher.

Jeder Schnittstelle liegt ein Protokoll zugrunde. Das Protokoll beschreibt die 
Vor- und Nachbedingungen der Schnittstellenelemente. Dazu gehören die erlaubten
Reihenfolgen, in denen Methoden der Schnittstelle aufgerufen werden dürfen, 
sowie Annahmen über Eingabeparameter und Zusicherungen über Ausgabeparameter. 
Das Protokoll von Modulen wird in der Modulsicht beschrieben.
Dort, wo es sinnvoll ist, sollte es mit Hilfe von Zustands- oder 
Sequenzdiagrammen spezifiziert werden. Diese sind dann einzusetzen, wenn der
Text allein kein ausreichendes Verständnis vermittelt (insbesondere bei 
komplexen oder nicht offensichtlichen Zusammenhängen).

Der Bezug zur konzeptionellen Sicht muss klar ersichtlich sein. Im Zweifel 
sollte er explizit erklärt werden. Auch für diese Sicht muss die Entstehung 
anhand der Strategien erläutert werden.}


%%%%%%%%%%%%%%%%%%%%%%%%%%%%%%%%%%%%%%%%%%%%%%%%%%%%%%%%%%%%%%%%%%%%%%%%
\section{Datensicht} \label{sec:datensicht}

{\itshape Hier wird das der Anwendung zugrundeliegende Datenmodell beschrieben. 
Hierzu werden neben einem erläuternden Text auch ein oder mehrere 
{UML}-Klassendiagramme verwendet. Das hier beschriebene Datenmodell wird u.\,a.\ 
jenes der Anforderungsspezifikation enthalten, allerdings mit 
implementierungsspezifischen Änderungen und Erweiterungen. Siehe die gesonderten
Hinweise.}


%%%%%%%%%%%%%%%%%%%%%%%%%%%%%%%%%%%%%%%%%%%%%%%%%%%%%%%%%%%%%%%%%%%%%%%%
\section{Ausführungssicht} \label{sec:ausfuehrung}

{\itshape Die Ausführungssicht beschreibt das Laufzeitverhalten. Hier werden die
Laufzeitelemente aufgeführt und beschrieben, welche Module sie zur Ausführung 
bringen. Ein Modul kann von mehreren Laufzeitelementen zur Laufzeit verwendet 
werden. Die Ausführungssicht beschreibt darüber hinaus, welche Laufzeitelemente 
spezifisch miteinander kommunizieren. Zudem wird bei verteilten Systemen 
(z.\,B.\ Client-Server-Systeme) dargestellt, welche Module von welchen Prozessen
auf welchen Rechnern ausgeführt werden.}


%%%%%%%%%%%%%%%%%%%%%%%%%%%%%%%%%%%%%%%%%%%%%%%%%%%%%%%%%%%%%%%%%%%%%%%%
\section{Zusammenhänge zwischen Anwendungsfällen und Architektur}
\sectionmark{Zusammenhänge AF u. Architektur} \label{sec:anwendungsfaelle}

{\itshape In diesem Abschnitt sollen Sequenzdiagramme mit Beschreibung(!) für 
\variante{zwei bis drei von Euch ausgewählte Anwendungsfälle}%
{einen von Euch ausgewählten Anwendungsfall}
erstellt werden. Ein Sequenzdiagramm beschreibt den Nachrichtenverkehr zwischen 
allen Modulen, die an der Realisierung des Anwendungsfalles beteiligt sind. 
\variante{Wählt die Anwendungsfälle so, dass nach Möglichkeit alle Module Eures
entworfenen Systems in mindestens einem Sequenzdiagramm vorkommen. Falls Euch 
das nicht gelingt, versucht möglichst viele und die wichtigsten Module 
abzudecken.}%
{Dazu könnt ihr Euch einen Anwendungsfall heraussuchen, der möglichst viele 
Module der  Architektur abdeckt. In SWP-2 werden wir mehrere Anwendungsfälle
betrachten und eine umfangreichere Abdeckung der Architektur anstreben.} }


%%%%%%%%%%%%%%%%%%%%%%%%%%%%%%%%%%%%%%%%%%%%%%%%%%%%%%%%%%%%%%%%%%%%%%%%
\section{Evolution} \label{sec:evolution}

{\itshape Beschreibt in diesem Abschnitt, welche Änderungen Ihr vornehmen müsst,
wenn sich Anforderungen oder Rahmenbedingungen ändern. Insbesondere würden 
hierbei die in der Anforderungsspezifikation unter \glqq{}Ausblick\grqq{} 
genannten Punkte behandelt werden.}

\dots

\textbf{Creative-Mode}
{
Der Creative-Mode soll ein eigener Solo-Modus werden, indem der Spieler sein Schiff nach freiem Belieben aus einer Liste aller im Spiel enthaltenen Eigenschaften für ein Schiff erstellen kann. Zunächst wählt der Spieler ein Raumschifflayout, anschließend bekommt er eine Auswahl an Systemen, eine Auswahl für das Level der Systeme, eine Auswahl an Waffen und deren Level und eine Auswahl an Crew, wobei man bei jedem Crewmitglied die Spezies (und die damit verbundene Spezialfähigkeit) und die Level individuell einstellen kann. 
Das Schiff kann im Creative Mode als (benannte) Vorlage gespeichert werden, um es bei zukünftigen Testspielen selbst oder als Gegner zu verwenden.

Hat man dann seine eigene Ausrüstung nach seinem Belieben ausgewählt, kann man damit nun ein Testspiel starten (Hierbei werden keine Errungenschaften o.Ä. freigeschaltet). Ein Testspiel verhält sich genau so wie ein normales Spiel, nur dass man nichts freischalten kann und somit wird sich dieses Testspiel, wie auch alles andere im Creative-Mode, nicht auf den normalen Spielfortschritt auswirken. 

Alternativ kann man auch einen Testkampf (gegen einen zufälligen, ähnlich angepassten Gegner oder gegen einen selbst konfigurierten Gegner) starten. Hier wird man also in einen Kampf Feld gespawnt, kann sich auf den Kampf vorbereiten und hat dann Buttons, mit dem man das Gegnerische Schiffe spawnen kann. Man kann nach eigenem Belieben ein oder mehrere Schiffe spawnen, diese aus Vorlagen oder zufälligen Vorlagen spawnen, und man kann für jeden Gegner eine Schwierigkeit wählen. Sobald man keine Lust mehr hat, beendet man den Testkampf einfach über einen Button. 

\textit{Sinn des Modus:}
Man kann neue Setups testen, auf die man hin arbeiten will, oder man lässt seinen Frust an vergleichsweise schlechten Gegnern aus, welche man mit viel zu übertriebenen Waffen zerschmettert. Wie der Name schon sagt, soll der Kreativität des Spielers keine Grenzen gesetzt werden.  

}



\end{document}

%%% Local Variables: 
%%% mode: latex
%%% mode: reftex
%%% mode: flyspell
%%% ispell-local-dictionary: "de_DE"
%%% TeX-master: t
%%% End: 
